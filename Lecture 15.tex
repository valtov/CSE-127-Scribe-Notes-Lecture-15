% !TEX TS-program = pdflatex
% !TEX encoding = UTF-8 Unicode

% This is a simple template for a LaTeX document using the "article" class.
% See "book", "report", "letter" for other types of document.

\documentclass[11pt]{article} % use larger type; default would be 10pt

\usepackage[utf8]{inputenc} % set input encoding (not needed with XeLaTeX)

%%% Examples of Article customizations
% These packages are optional, depending whether you want the features they provide.
% See the LaTeX Companion or other references for full information.

%%% PAGE DIMENSIONS
\usepackage{geometry} % to change the page dimensions
\geometry{a4paper} % or letterpaper (US) or a5paper or....
% \geometry{margin=2in} % for example, change the margins to 2 inches all round
% \geometry{landscape} % set up the page for landscape
%   read geometry.pdf for detailed page layout information

\usepackage{graphicx} % support the \includegraphics command and options

% \usepackage[parfill]{parskip} % Activate to begin paragraphs with an empty line rather than an indent

%%% PACKAGES
\usepackage{booktabs} % for much better looking tables
\usepackage{array} % for better arrays (eg matrices) in maths
\usepackage{paralist} % very flexible & customisable lists (eg. enumerate/itemize, etc.)
\usepackage{verbatim} % adds environment for commenting out blocks of text & for better verbatim
\usepackage{subfig} % make it possible to include more than one captioned figure/table in a single float
% These packages are all incorporated in the memoir class to one degree or another...

%%% HEADERS & FOOTERS
\usepackage{fancyhdr} % This should be set AFTER setting up the page geometry
\pagestyle{fancy} % options: empty , plain , fancy
\renewcommand{\headrulewidth}{0pt} % customise the layout...
\lhead{}\chead{}\rhead{}
\lfoot{}\cfoot{\thepage}\rfoot{}

%%% SECTION TITLE APPEARANCE
\usepackage{sectsty}
\allsectionsfont{\sffamily\mdseries\upshape} % (See the fntguide.pdf for font help)
% (This matches ConTeXt defaults)

%%% ToC (table of contents) APPEARANCE
\usepackage[nottoc,notlof,notlot]{tocbibind} % Put the bibliography in the ToC
\usepackage[titles,subfigure]{tocloft} % Alter the style of the Table of Contents
\renewcommand{\cftsecfont}{\rmfamily\mdseries\upshape}
\renewcommand{\cftsecpagefont}{\rmfamily\mdseries\upshape} % No bold!

%%% END Article customizations

%%% The "real" document content comes below...

\title{Lecture 15}
\author{TLS and secure channels}
%\date{} % Activate to display a given date or no date (if empty),
         % otherwise the current date is printed 

\begin{document}
\maketitle

\section{Common Crytographic Network Protocols}

\subsection{TLS (Transport Layer Security)}
\begin{itemize}
  \item  Used to provide an encryption wrapper around HTTP to make HTTPS, and for
  many other application layer protocols.
  \item TLS is wrapped around the application layer.
  \item Security goals: Authenticate server, confidentiality and integrity of
  traffic. Ensure that client is connected to the server they think they are
  connected to.
  \item Originally called the Secure Socket Layer (SSL).
\end{itemize}

\subsection{SSH (Secure Shell)}
\begin{itemize}
  \item Encrypted alternative to Telnet, Telnet uses an unencrypted connection
  that sends information in plaintext, which is easily exploitable.
  \item Security goals: Authenticate server and client, confidentiality and
  integrity of traffic.
\end{itemize}

\subsection{IPsec (Internet Protocol Security)}
\begin{itemize}
  \item Provides an encrypted, authenticated alternative to IP. Complicated set
  of protocols which attempt to replace the IP layer.
  \item Regular IP is insecure because anyone can view the payload of packets.
  \item Tunnel IPSec through IP.
  \item Commonly used for VPNs (Virtual Private Networks).
  \item Security goals: client and server authentication, authenticate headers,
  optionally encrypt headers, ensure confidentiality and integrity of payloads.
\end{itemize}

\section{Constructing a Secure Encrypted Channel}
To construct a secure encrypted channel there are several steps of prelinary
communication that the client and server must perform before the channel can be
established.

\subsection{Encrypt and MAC data}

\includegraphics[scale=.7]{./tls1.png}

{\parindent0pt Alice[left] and Bob[right] want to communicate on a secure 
channel on that is protected against passive easedroppers and man-in-the-middle
attacks.}

\bigskip
{\parindent0pt Assuming Alice and Bob have shared a set of keys, Alice sends her
AES ciphertext and the MAC of the ciphertext to Bob.  Bob can now check the MAC
and decrypt the cipher text to get the original message.}

\newpage
\subsection{Diffie-Hellman key exchange}
In order to negotiate charing encryption and MAC keys, there must be a
Diffie-Hellman key exchange.

\includegraphics[scale=.7]{./tls2.png}

{\parindent0pt \textbf{NOTE:} In the image above $g^{a}$  should be read as 
$g^{a}\ mod\ p$ and likewise $g^{b}$ should be read $g^{b}\ mod\ p$}

\bigskip
{\parindent0pt If a Diffie-Hellman key exchange has occurred then both Alice 
and Bob will have a \textbf{shared secret}. In this case Alice and Bob's shared
secret is $g^{ab}$.}

\bigskip
{\parindent0pt Using this shared secret Bob and Alice can use a Key Derivation
Function (KDF), 
\smallskip
$k_e, k_m\ =\ KDF(g^{ab})$, which we can think of as a hash function that is 
used to create encryption and MAC keys they can use for symmetric crypto.

\newpage
\subsection{Ensure Authenticity of Endpoints}
There is a vulnerabiliy with this approach because if there is an active
man-in-the-middle attack, they could man-in-the-middle attack the Diffie-Hellman
key exchange.  To ensure the authenticity of the endpoint we must use
\textbf{digital signatures}, to prevent man-in-the-middle interference of the
key exchange.
\smallskip
You can either have one or both parties sign the key exchange with a long-term
public key.

\includegraphics[scale=.7]{./tls3.png}

For the sake of example, assume that Alice is the client [web browser], Bob is 
the web server, Alice knows the server's long-term key, and Alice is trying to
communicate with Bob.

\bigskip
To protect the Diffie-Hellman key exchange Bob will sign the key exchange with 
his long-term key.  Since Alice knows both Bob's original long term-key and the
signature which Bob has given Alice can verify the signature by using Bob's
public key before progressing with the encryption.

\newpage
\subsection{Trusting Signatures}
While we may now be protected against man-in-the-middle attacks targeted towards
the Diffie-Hellman key exchange, we have still not verified the integrity of
Bob's public signing key which Alice has received.

\includegraphics[scale=.7]{./tls4.png}

The public signing key that Bob sends to Alice is susceptible to
active man-in-the-middle attacks, so we must determine a way to trust the keys 
the client receives.

\bigskip
Assuming that there was a man-in-the-middle who was overseeing all the
communications between Alice and Bob, an attacker could substitute Bob's public
key with their own generated public key and Alice would not be able to tell 
the difference.

\bigskip
Due to the nature of this attack, we must have an \textbf{external} method of
establishing trust in keys.

\newpage
\subsection{Establishing Trust in Keys}
TODO

\section{TSL: Transport Layer Security}

\subsection{TSL Overview}

\subsection{TLS 1.2 with Diffie-Hellman Key Exchange}

\subsubsection{Step 1}

\subsubsection{Step 2}

\subsubsection{Step 3}

\subsubsection{Step 4}

\subsubsection{Step 5}

\subsubsection{Step 6}

\subsubsection{Step 7}

\subsubsection{Step 8}

\subsection{Certificates and Certificate Authorities in TLS}

\subsection{Revocation}

\subsection{Root CAs on OS X}

\subsection{Man-in-the-Middle Attack Using Rogue Cert}

\subsection{TLS 1.2 with RSA Key Exchange}

\subsection{How TLS Achieves Its Security Goals}

\subsection{What If a Private Key Gets Stolen or Compromised?}

\subsection{TLS v. 1.2 and Below Vulnerabilities}

\subsection{TLS 1.3}

\subsection{TLS Key Theft and Other Risks in the Wild (not in lecture)}

\subsection{The “Crypto Wars” and the Historical Development of TLS (not in lecture)}

\end{document}
